%!TEX root = ../article/Journal_SelfCalibration.tex

\section{Discussion}
\label{sec:conclusions}

This paper presented an online study where eight subjects controlled a cursor using EEG error-related potentials with a self calibration of the decoder. All the subjects performed an online experiment, where the task (i.e. reaching a target position) and the classification parameters were learned in parallel, in an unsupervised way and transparent to the user. The results showed how both task and classifier could be learned, outperforming standard calibration times in terms of whole execution time of the experiment.
%

%Calibration is an essential component of most BCI systems and has rec
The calibration phase is indeed one of the main limitations of current BCI approaches, as they strongly limit the usability of these systems and their applications in out-of-the-lab conditions \cite{millan10}. Despite recent approaches having demonstrated that it is possible to greatly reduce the calibration time, only a few works have succeeded in removing this phase. As stated in the introduction, Kindermans et al. exploited the multiple repetitions of P300 stimuli and context information of the task to have a zero-calibration P300 approach \cite{Kindermans2012a,kindermans2014true}. Specifically, they exploited different particularities of a P300 speller: first, that among the multiple stimulations, only one event out of six encoded a P300 potential in the speller paradigm (i.e. a fixed P300 ratio); second, the multiple repetitions needed to increase the signal-to-noise ratio of the signal; and third, the ability of rolling back and re-learning classifiers thanks to language models. Indeed, in the P300 speller problem only one row and one column should elicit a P300 response when flashed. Compared with their approach, we could not exploit these constraints, and thus our only constraint was that of having a finite number of reaching positions. As a result, we also needed to add an intelligent planning and exploration algorithm whose objective was to acquire data that maximized the difference between confounded targets (see \cite{grizou2014calibration, grizou2014interactive} for details on the exploration algorithms and implementation). 

% vs standard approach
Comparisons with the standard calibration procedure illustrate the advantages of such an adaptive system. First, it is possible to start the operation of the device from the very beginning, allowing to perform the task more efficiently (see Table \ref{tab:onlineXPsummary}). Second, the obtained accuracies are as good than those obtained with a calibration procedure (see Figs. \ref{fig:labels} and \ref{fig:std_vs_sc}). Therefore, there is not an expected loss of performance in the long term. Third, there is no need for expert supervision or ad-hoc stopping criterion during the self-calibration process (e.g. number of required trials, time or performance constraints) as the system adapts automatically to the user specific behavior. 

Notice, however, that an important difference between a standard calibration and self-calibration is the uncontrolled ratio of error vs correct trials. Indeed, it is common in the literature that ErrP calibrations follow fixed ratios between 20\% or 30\% \cite{chavarriaga2010learning, iturrate13}. Even though the use of ErrP in practical applications cannot assure a stationary percentage of errors, this ratio has empirically shown to generate more distinguishable EEG patterns and  better accuracies \cite{chavarriaga2010learning}. As our planning method cannot guarantee the same agent behavior than during a typical calibration phase, the quality of the signals generated by the users might be affected, with strong variations in the error rate from random behaviors (80\% of errors) to close to optimal behaviors (below 10\%), see Fig. \ref{fig:percentage_errors}. Nonetheless, as shown in Fig. \ref{fig:std_vs_sc}, we did not find any substantial differences neither at the signal level nor at the classification level between standard calibration (with a fixed rate of 30\%) and our proposed self-calibration. However, these results have to be taken carefully, as the number of subjects was not enough to draw any definite conclusion and thus further tests will be required.

Although the unsupervised approach attains similar results in terms of accuracy, and insignificantly different grand average signals when compared to those obtained using supervised calibration (see Fig. \ref{fig:std_vs_sc}), simulations showed differences in performance namely between the number of incorrect targets (see Table \ref{tab:onlineXPsummary}). There are several plausible explanations for the larger number of wrong targets during self-calibration. First, initial targets do not use a stable classifier and, consequently, it is more probable to misclassify some signals leading to a different target from the intended one. Second, self-calibration also reaches more targets and has more chances to fail. Third, there is no difference between calibration and control phases which may also induce some confusion on the users since the device sometimes aims to reduce its uncertainty on the signals instead of progressing towards the goal. Future work is required to understand the impact of each of these possible reasons and, if possible, to derive a strategy that minimizes wrong targets.

Another influential limitation of the proposed approach is that, with the current algorithm implementation, the users cannot (or should not) switch the desired target during one reaching operation. Under this circumstance, the system could be affected in two ways depending on the relative distance between targets or number of switches among others: first, it could increase the convergence time exponentially; and second, it may severely affect the labeling quality obtained once the system converges to one target. As a possible solution, a target reset function could be implemented by explicitly modeling two possible target locations instead of one. Nonetheless, further work will be needed to understand the impact that this target switches may have on the proposed system and its usability.

An important aspect of the proposed approach is that not only it serves as a calibration system from scratch, but may serve as well as a procedure that can constantly adapt to the user signals reducing the effect of possible non-stationarities. Thus, the proposed approach could serve as a complement rather than an alternative to standard calibration approaches. For instance, the self-calibration algorithm could be run after a standard calibration of few minutes where the user gets accustomed to the protocol and data can be recorded in a supervised manner. In fact, the results showed how several subjects (bottom left subjects in Fig. \ref{fig:labels}b) were affected by the self-calibration, and thus might have benefited from a supervised warm-up period.
%Interestingly, these subjects were naive BCI users and thus might have needed a warm-up period prior to the protocol execution.

An avenue for future research is how to exploit constraints to develop self-calibration BCIs for other tasks or brain signals. 	On one hand and regarding the control signals, there have been numerous works in the invasive and non-invasive community showing how it is possible to control devices for reaching tasks (e.g. motor rhythms \cite{mcfarland2010electroencephalographic}, slow-cortical potentials \cite{birbaumer1999spelling} or electrocorticography \cite{schalk2008two}). We postulate that, as long as there is a finite number of possible targets to reach, our self-calibration approach could be applicable. This idea was partially followed by the invasive community \cite{Orsborn12, orsborn2014closed}, where they studied how to minimize calibration in a center-out reaching task assuming that reaching a target should be done in a straight line. We believe that it would be possible to further extend the algorithm to deal with unknown targets in such scenarios. 
%
On the other hand, the use of our proposed self-calibration approach on other BCI tasks solely depends on the existence of task constraints that can be used to model all the possible outcomes of the executed task.
%
These further studies may confirm the usability of the proposed self-calibration approach, improving this way the usability of BCI systems for out-of-the-lab applications.

%\todo{Could we open on the application of this work to other fields? : Finally, we believe the same algortihmic principle could be applied to other human-machine interaction systems using different modalities of interaction such as speech or gestures.}