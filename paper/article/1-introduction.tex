%!TEX root = ../article/Journal_SelfCalibration.tex

\section{Introduction}
\label{sec:intro}

The field of brain-computer interfaces (BCI) has witnessed largely successful applications in different contexts such as the control of assistive devices or communication restoration \cite{millan10}. However, their impact has been limited to in-lab applications due to intrinsic limitations of the technology. The brain signals used in BCI have a high variability caused by factors such as non-stationarities \cite{vidaurre2010towards}, task-dependent variations \cite{Iturrate2013task	}, and large subject specificity \cite{blankertz2010single}. To solve these problems, most applications rely on an initial calibration phase that needs to be frequently repeated due to the changing nature of the brain signals. This calibration process is a tedious and time consuming part in most BCI applications that hinders their deployment. First, it impedes an out-of-the-box use of the BCI both in assistive or leisure scenarios. Second, it increases the cost in professional and medical applications such as those carried out in hospitals. Third,  calibration is an overload that affects the usability of the system in the long term and, therefore, the introduction and acceptance of such systems.

Previous works have aimed at minimizing and/or simplifying the calibration phase. For instance, in motor imagery \cite{millan10}, the supervised and unsupervised adaptation of classification parameters has shown to avoid inter-session recalibration \cite{vidaurre11} and to increase the BCI performance \cite{vidaurre11b}. Other approaches have tried to limit the calibration time by using clustering methods in the feature space \cite{krauledat2006reducing}; ensembles of classifiers built from very large datasets \cite{fazli2009subject}; inter-subject generalization by using a pool of subjects to initialize the classifier \cite{iturrate2011minimizing, faller2012autocalibration, lotte2010learning}; or even by generating artificial training datasets via resampling \cite{lotte:inria-00599325}.

Despite their successful results, these methods aim at reducing the calibration phase rather than completely removing it. On the other hand, free-calibration BCIs (i.e. \textit{plug-and-play BCIs}) remain scarce as they need to achieve BCI control together with a real-time, transparent, and unsupervised calibration executed in parallel. For these approaches, the main idea is to exploit redundancy or task constraints to limit the possible space of brain signals decoders to a tractable one. In non-invasive BCI, Kindermans et al. have thoroughly shown that such kind of BCI is feasible for P300 spellers \cite{Kindermans2012a,kindermans2014true}. In this approach, the self-calibration speller exploits the multiple repetitions of P300 stimuli and context information of the task, namely word constraints and grammar rules. Similarly, Orsborn et al. have also achieved control from scratch for reaching tasks in an invasive brain-machine interface \cite{Orsborn12, orsborn2014closed}. They initialized the decoder to a random behavior and updated it during online control using the assumption that the targets should only be reached following a straight line.

%\todo{regarding PY comments, here is a draft: Our approach differs form the work of Kindermans et al. and Orsborn et al. in that it does not require any particular constraint on the task. For example, in the work of Kindermans et al. the ratio of positive to negative P300 signals is forced to be unbalanced, which helps identifying which cluster of signals is associated to a positive or a negative response. Moreover our work generalizes to any sequencial problem represented as an MDP.}

This paper presents a new approach and an online study of a self-calibration method for BCI control applications based on error potentials \cite{chavarriaga2010learning, iturrate13}. The key insight of this protocol is that the task to be solved provides constraints that can be exploited to simultaneously calibrate, in an unsupervised fashion, the BCI while controlling the device. With respect to our preliminary works, the current study proposes an alternative algorithm that does not suffer from the limitations of \cite{grizou2014calibration}, namely failures due to overconfident estimates. Furthermore, it shows the applicability to online scenarios of the planning techniques described and evaluated in \cite{grizou2014interactive}. 
%
The method has been evaluated online in a reaching task over a discrete grid. Results of eight subjects show that the performance of the system was above that obtained using a standard BCI calibration method in terms of time required to reach the intended goal and in the number of goals achieved for a fixed amount of time. Furthermore, after a certain time, the system is indeed calibrated without incurring any overhead in the long term.

%\todo{Should we include planning and other algorithmic aspects here?}

%\todo{Reaching task: The user control the movements of a device in a N dimensional space to reach a specific target position. A goal-reached action can be performed to indicate the targeted position has been reached, in practice the agent does not move and signals it took an action by [displaying circles around him].}